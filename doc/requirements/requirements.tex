\documentclass{article}

\usepackage{hyperref}
\usepackage{titling}

\setlength{\droptitle}{-25mm}
\title{Requirements For Virtuous Hummingbird}
\author{\href{https://github.com/wintertideheir}{wintertideheir}}

\begin{document}

\maketitle
\tableofcontents

\section{Introduction}

Virtuous Hummingbird is a program for the organization and advancement of personal virtue.
By virtue, we mean the virtues of virtue ethics, personal qualities indicating a pattern of repeated behavior.
These qualities are related causally to form a hierarchical tree of virtues. 
The simple nature of virtues and virtue trees makes virtue ethics an unusually intuitive and practical ethical theory.
By presenting a way to create, edit, visualize, and track these virtues, Virtuous Hummingbird facilitates living a virtuous life.

\section{Project Information}

\subsection{Background}

Virtuous Hummingbird was concieved as an improvement on another concept-mapping system, Cmap, for mapping virtue ethics.

Cmap is a trademark applied to a series of products created by the Florida Institute for Human and Machine Cognition.
Cmap products are intended to empower users to represent knowledge as concept maps.
CmapTools, a Cmap product, is a desktop application that allows users to view and manipulate concept maps.
Because Cmap products could represent any knowledge, CMapTools was adopted as a tool to research virtue ethics.

However, Cmap products are only optimized for general use.
When faced with high volume graphs in virtue ethics, the representations they unnavigatable and unusable.
CmapTools is also limited in it's ability to parse concept maps and generate new insights.
In the long term, CmapTools is not a programs that can be easily extended to perform other functions.

A new kind of concept-mapping software was concieved, optimized for virtue ethics and built with domain-specific utilities.
To avoid information overload, the concept map would be represented as simple nodes defined only be color and position.
The information associated with these nodes would be specific to the virtues of virtue ethics.
Finally, multiple utilities would be built around this base.

\subsection{Purpose}

% What problem will this product solve?

\subsection{Objectives}

% What concrete goals must this products achieve?

\subsection{Metrics}

% How do we measure success?

\subsection{Summary}

% Summarize who has what problem, and how the product solves it.

\section{Auxiliary}

\subsection{Risks}

% Summarize the potential risks involved in pursuing this project.

\subsection{Information}

% What do we know for certain and what assumptions do we make?

\section{Context}

\subsection{Demographics}

% What do we know about our users?

\subsection{Comparables}

The products that offer the closest functionality to Virtuous Hummingbird are task trackers.
Task trackers expand on the basic task-list with features such as categories, tags, and prioritization.
Their primary advantage is their simplicity of use.
Lists are very simple to create, edit, and view.
Their primary disadvantage is their difficulty in scaling as the number and variety of tasks grows.
Task-trackers attempt to mitigate their scaling problems with organization tools, requiring at least a little planning.
There are several popular task-trackers available as of January 2021:

\begin{description}
    \item[Google Tasks], free.
    A widely used product on account of it being bundled with the standard apps for a Google account.
    It provides the ability to create sub-tasks, set due dates, and send notifications.
    It integrates with other Google apps, namely Gmail and Google Calendar.
    \item[Microsoft To Do], from \$5 per user per month.
    This product appears to offer similar features to Google Tasks.
    \item[Remember The Milk], free or \$40 per year.
    A simple task-list service with a focus on everyday, non-business application.
    Features an advanced auto-complete feature that makes creating tasks convenient.
    It can use e-mail, text, Twitter, and other apps.
    It can connect files from Google Drive and Dropbox to tasks.
    It features a smart search feature and priority system.
    \item[Trello], free to \$17.5 per user per month.
    A professional-focused task tracker.
    Based around dividing tasks into teams.
    These tasks can be detailed with comments, attachments, and due dates, and then viewed with an sophisticated interface.
    Programming and automation is available to streamline that process as well.
    \item[Todoist], free to \$5 per user per month.
    A task-tracker with many interesting features.
    It's core feature are well-detailed tasks.
    It also allows collaboration on tasks, including delegation and comments.
    Gamification is present as well, with a karma system and productivity visualizations.
    \item[Asana], free to per contract pricing.
    A feature-risk task-tracker with many different ways of viewing a task list, including a time-line view, a table view, a multi-list view, etc.
    It also features programming / automation to streamline task creation and management.
\end{description}

Another class of products to consider are time-tracking apps.
Rather than planning an activity in advance, time-tracking apps record an activity as it occurs.
They often have similar features to task-trackers, as well as billing, export, and retrospective editing features.
There are several popular time-tracking apps, as of January 2021:

\begin{description}
    \item[Toggl Track], free or \$9 per user per month.
    An easy-to-use tracking app.
    Activities can be tracked with or without a name, project, or tags, and then edited later.
    It also integrates very well with Google apps and works on many different platforms.
    \item[Harvest], free or \$12 per user per month.
    A team focused time-tracker.
    Teams can connect their logs to an administration log.
    It can also integrate well with task-trackers and other apps.
    \item[Everhour], free or \$5 per user per month.
    A light-weight time-tracker.
    It integrates well with other productivity apps and offers employee integration.
\end{description}

Calendar applications often provide very simple versions of the functionality mentioned above.
Full capability is provided only in conjunction with the above apps.

\subsection{Major Features}

% What major features should our product have?

\subsection{Priorities}

% What features and requirements are more important?

\subsection{Limitations}

% What limitations does our product have?

\subsection{Resources}

% What resources do we have to create this product?

\subsection{Version Plan}

% In what order should we implement features?

\end{document}
