\documentclass{article}

\title{Business Requirements}

\begin{document}
    
\maketitle

\section{Business Requirements}

\subsection{Background}

Virtuous Hummingbird was concieved as an improvement on another concept-mapping system, Cmap, for mapping virtue ethics.

Cmap is a trademark applied to a series of products created by the Florida Institute for Human and Machine Cognition.
Cmap products are intended to empower users to represent knowledge as concept maps.
CmapTools, a Cmap product, is a desktop application that allows users to view and manipulate concept maps.
Because Cmap products could represent any knowledge, CMapTools was adopted as a tool to research virtue ethics.

However, Cmap products are only optimized for general use.
When faced with high volume graphs in virtue ethics, the representations they unnavigatable and unusable.
CmapTools is also limited in it's ability to parse concept maps and generate new insights.
In the long term, CmapTools is not a programs that can be easily extended to perform other functions.

A new kind of concept-mapping software was concieved, optimized for virtue ethics and built with domain-specific utilities.
To avoid information overload, the concept map would be represented as simple nodes defined only be color and position.
The information associated with these nodes would be specific to the virtues of virtue ethics.
Finally, multiple utilities would be built around this base.

\subsection{Business Opportunity}

\subsection{Business Objectives}

\subsection{Success Metrics}

\subsection{Vision Statement}

\subsection{Business Risks}

\subsection{Assumptions and Dependencies}

\section{Scope and Limitations}

\subsection{Major Features}

\subsection{Scope of Initial Release}

\subsection{Scope of Subsequent Releases}

\subsection{Limitations and Exclusions}

\section{Business Context}

\subsection{Stakeholder profiles}

\subsection{Project priorities}

\subsection{Deployment Considerations}

\end{document}